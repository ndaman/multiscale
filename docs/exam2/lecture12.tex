% Options for packages loaded elsewhere
\PassOptionsToPackage{unicode}{hyperref}
\PassOptionsToPackage{hyphens}{url}
%
\documentclass[
  letterpaper,
  ignorenonframetext,
  aspectratio=43,
  handout,
  12pt]{beamer}
\usepackage{pgfpages}
\setbeamertemplate{caption}[numbered]
\setbeamertemplate{caption label separator}{: }
\setbeamercolor{caption name}{fg=normal text.fg}
\beamertemplatenavigationsymbolsempty
% Prevent slide breaks in the middle of a paragraph
\widowpenalties 1 10000
\raggedbottom
\setbeamertemplate{part page}{
  \centering
  \begin{beamercolorbox}[sep=16pt,center]{part title}
    \usebeamerfont{part title}\insertpart\par
  \end{beamercolorbox}
}
\setbeamertemplate{section page}{
  \centering
  \begin{beamercolorbox}[sep=12pt,center]{part title}
    \usebeamerfont{section title}\insertsection\par
  \end{beamercolorbox}
}
\setbeamertemplate{subsection page}{
  \centering
  \begin{beamercolorbox}[sep=8pt,center]{part title}
    \usebeamerfont{subsection title}\insertsubsection\par
  \end{beamercolorbox}
}
\AtBeginPart{
  \frame{\partpage}
}
\AtBeginSection{
  \ifbibliography
  \else
    \frame{\sectionpage}
  \fi
}
\AtBeginSubsection{
  \frame{\subsectionpage}
}
\usepackage{amsmath,amssymb}
\usepackage{lmodern}
\usepackage{ifxetex,ifluatex}
\ifnum 0\ifxetex 1\fi\ifluatex 1\fi=0 % if pdftex
  \usepackage[T1]{fontenc}
  \usepackage[utf8]{inputenc}
  \usepackage{textcomp} % provide euro and other symbols
\else % if luatex or xetex
  \usepackage{unicode-math}
  \defaultfontfeatures{Scale=MatchLowercase}
  \defaultfontfeatures[\rmfamily]{Ligatures=TeX,Scale=1}
\fi
\usetheme[]{metropolis}
% Use upquote if available, for straight quotes in verbatim environments
\IfFileExists{upquote.sty}{\usepackage{upquote}}{}
\IfFileExists{microtype.sty}{% use microtype if available
  \usepackage[]{microtype}
  \UseMicrotypeSet[protrusion]{basicmath} % disable protrusion for tt fonts
}{}
\makeatletter
\@ifundefined{KOMAClassName}{% if non-KOMA class
  \IfFileExists{parskip.sty}{%
    \usepackage{parskip}
  }{% else
    \setlength{\parindent}{0pt}
    \setlength{\parskip}{6pt plus 2pt minus 1pt}}
}{% if KOMA class
  \KOMAoptions{parskip=half}}
\makeatother
\usepackage{xcolor}
\IfFileExists{xurl.sty}{\usepackage{xurl}}{} % add URL line breaks if available
\IfFileExists{bookmark.sty}{\usepackage{bookmark}}{\usepackage{hyperref}}
\hypersetup{
  hidelinks,
  pdfcreator={LaTeX via pandoc}}
\urlstyle{same} % disable monospaced font for URLs
\newif\ifbibliography
% Make links footnotes instead of hotlinks:
\DeclareRobustCommand{\href}[2]{#2\footnote{\url{#1}}}
\setlength{\emergencystretch}{3em} % prevent overfull lines
\providecommand{\tightlist}{%
  \setlength{\itemsep}{0pt}\setlength{\parskip}{0pt}}
\setcounter{secnumdepth}{-\maxdimen} % remove section numbering
\usepackage{pgfpages}
\pgfpagesuselayout{2 on 1}
\providecommand{\tightlist}{%
\setlength{\itemsep}{0pt}\setlength{\parskip}{0pt}}
\makeatletter
\makeatother
\let\Oldincludegraphics\includegraphics
\renewcommand{\includegraphics}[2][]{\Oldincludegraphics[width=\textwidth,height=0.7\textheight,keepaspectratio]{#2}}
\ifluatex
  \usepackage{selnolig}  % disable illegal ligatures
\fi

\author{}
\date{}

\begin{document}

\begin{frame}
Lecture 12 - Hashin-Shtrikman Bounds

Dr.~Nicholas Smith

Wichita State University, Department of Aerospace Engineering

March 18, 2021
\end{frame}

\begin{frame}{schedule}
\protect\hypertarget{schedule}{}
\begin{itemize}
\tightlist
\item
  Mar 18 - Hashin-Shtrickman bounds
\item
  Mar 23 - Periodic Boundary Conditions
\item
  Mar 25 - Fourier Analysis
\item
  Mar 30 - Method of Cells
\end{itemize}
\end{frame}

\begin{frame}{outline}
\protect\hypertarget{outline}{}
\begin{itemize}
\tightlist
\item
  hashin-shtrikman
\item
  boundary conditions
\end{itemize}
\end{frame}

\hypertarget{hashin-shtrikman}{%
\section{hashin-shtrikman}\label{hashin-shtrikman}}

\begin{frame}{bounds}
\protect\hypertarget{bounds}{}
\begin{itemize}
\tightlist
\item
  We consider the Voigt and Reuss micromechanics models as bounding
  cases, properties should need exceed the limits of these two cases
\item
  Hashin and Shtrikman used variational principles to define more
  rigorous bounds for composite properties
\item
  They did this by comparing a heterogeneous composite RVE with an
  equivalent homogeneous RVE
\end{itemize}
\end{frame}

\begin{frame}{heterogeneous}
\protect\hypertarget{heterogeneous}{}
\[\begin{aligned}
  \sigma_{ij,j} &= 0\\
  \sigma_{ij} &= C_{ijkl} \epsilon_{kl}\\
  U = \frac{1}{2} C_{ijkl} \epsilon_{ij}\epsilon_{kl}
\end{aligned}\]
\end{frame}

\begin{frame}{homogeneous}
\protect\hypertarget{homogeneous}{}
\[\begin{aligned}
        \sigma_{ij,j}^{(0)} &= 0\\
        \sigma_{ij}^{(0)} &= C_{ijkl}^{(0)} \epsilon_{kl}^{(0)}\\
        U = \frac{1}{2} C_{ijkl}^{(0)} \epsilon_{ij}^{(0)}\epsilon_{kl}^{(0)}
\end{aligned}\]
\end{frame}

\begin{frame}{relation}
\protect\hypertarget{relation}{}
\begin{itemize}
\tightlist
\item
  To relate the two boundary problems, we introduce the following
\end{itemize}

\[\begin{aligned}
   u_i &= u_i^{(0)} + u_i^d\\
   \epsilon_{ij} &= \epsilon_{ij}^{(0)} + \epsilon_{ij}^d\\
   \sigma_{ij} &= p_{ij} + C_{ijkl}^{(0)} \epsilon_{kl} = p_{ij} + C_{ijkl}^{(0)}(\epsilon_{ij}^{(0)} + \epsilon_{ij}^d)
\end{aligned}\]

\begin{itemize}
\tightlist
\item
  \(u_i^d\) is the disturbance displacement field and \(p_{ij}\) is
  called the polarization stress
\end{itemize}
\end{frame}

\begin{frame}{boundary conditions}
\protect\hypertarget{boundary-conditions}{}
\begin{itemize}
\tightlist
\item
  One common RVE boundary condition is known as homogeneous displacement
\item
  Under homogeneous displacement boundary conditions we have
\end{itemize}

\[u_i = \bar{u}_i = u_i^{(0)}\]

along the boundary - Under this condition we have \(u_d=0\) along the
boundary
\end{frame}

\begin{frame}{hashin-shtrikman}
\protect\hypertarget{hashin-shtrikman-1}{}
\begin{itemize}
\tightlist
\item
  Hashin-Shtrikman then considered the following functional
\end{itemize}

\[ \Pi = \int_V (C_{ijkl}^{(0)}\epsilon_{ij}^{(0)}\epsilon_{kl}^{(0)} - \Delta C_{ijkl}^{-1}p_{ij}p_{kl} + p_{ij}\epsilon_{ij}^d + 2p_{ij}\epsilon_{ij}^{(0)})dV
\]

\begin{itemize}
\tightlist
\item
  Where
\end{itemize}

\[\begin{aligned}
  \Delta C_{ijkl} &= C_{ijkl} - C^{(0)}_{ijkl}\\
  p_{ij} &= \Delta C_{ijkl} \epsilon_{kl}\\
  \epsilon_{ij}^d &= \epsilon_{ij} - \epsilon_{ij}^{(0)}
\end{aligned}\]

\begin{itemize}
\tightlist
\item
  This functional corresponds to the strain energy in a composite when
  the strain field and polarization field are exact solutions
\end{itemize}
\end{frame}

\begin{frame}{hashin-shtrikman}
\protect\hypertarget{hashin-shtrikman-2}{}
\begin{itemize}
\tightlist
\item
  We can choose the comparison solid such that \(\delta \Pi\) will
  either be a local maximum or a local minimum
\item
  When \(\Delta C\) is negative definite then the stationary value of
  the functional is a minimum
\item
  When \(\Delta C\) is positive definite then the stationary value of
  the functional is a maximum
\item
  The functional will be stationary when
\end{itemize}

\[ \left ( C_{ijkl}^{(0)} \epsilon^d_{kl}\right)_{,j} + p_{ij,j} = 0 \]
\end{frame}

\begin{frame}{hashin-shtrikman}
\protect\hypertarget{hashin-shtrikman-3}{}
\begin{itemize}
\tightlist
\item
  In general, I don't know how often you will need to use the
  Hashin-Shtrikman bounds
\item
  For a more complete derivation, see textbook pp.~170-186
\end{itemize}
\end{frame}

\hypertarget{boundary-conditions-1}{%
\section{boundary conditions}\label{boundary-conditions-1}}

\begin{frame}{macro and micro fields}
\protect\hypertarget{macro-and-micro-fields}{}
\begin{itemize}
\tightlist
\item
  In micromechanics, one of our primary goals is to relate a
  heterogeneous material to some equivalent homogeneous material
\item
  We call \(\epsilon_{ij}\) and \(\sigma_{ij}\) the point-wise or
  microscopic strain and stress
\item
  \(\bar{\epsilon}_{ij}\) and \(\bar{\sigma}_{ij}\) are the macroscopic
  strain and stress, and are related by some unknown homogenized
  stiffness
\end{itemize}

\[\bar{\sigma}_{ij} = C_{ijkl}^\* \bar{\epsilon}_{kl}\]

\begin{itemize}
\tightlist
\item
  In a homogeneous body (or equivalent homogeneous body),
  \(\bar{sigma}_{ij}\) and \(\bar{\epsilon}_{ij}\) will be constant
  throughout
\end{itemize}
\end{frame}

\begin{frame}{average stress theorem}
\protect\hypertarget{average-stress-theorem}{}
\begin{itemize}
\tightlist
\item
  In general the stress field \(\sigma_{ij}\) will not be constant in a
  heterogeneous body
\item
  If a heterogeneous body is subjected to homogeneous tractions with no
  body forces such that
\end{itemize}

\[t_i^0 = \bar{\sigma}_{ij}n_j\]

\begin{itemize}
\tightlist
\item
  And we find that
\end{itemize}

\[\langle \sigma_{ij} \rangle = \bar{\sigma}_{ij}\]
\end{frame}

\begin{frame}{average strain theorem}
\protect\hypertarget{average-strain-theorem}{}
\begin{itemize}
\tightlist
\item
  Similarly, in general the strain field, \(\epsilon_{ij}\) will not be
  constant in a heterogeneous body
\item
  If a heterogeneous body is subjected to a homogeneous displacement
  such that
\end{itemize}

\[u_i^0 = \bar{\epsilon_{ij}}x_j\]

\begin{itemize}
\tightlist
\item
  And we find that
\end{itemize}

\[\langle \epsilon_{ij} \rangle = \bar{\epsilon}_{ij}\]
\end{frame}

\begin{frame}{hill mandel macrohomogeneity condition}
\protect\hypertarget{hill-mandel-macrohomogeneity-condition}{}
\begin{itemize}
\tightlist
\item
  Hill and Mandel posed the question: Under what conditions will the
  average strain energy density of a heterogeneous body be equivalent
  equivalent to a homogeneous body?
\item
  In other words, they wanted show under what conditons
\end{itemize}

\[\langle \sigma_{ij} \epsilon_{ij} \rangle = \bar{\sigma}_{ij} \bar{\epsilon}_{ij}\]
\end{frame}

\begin{frame}{hill mandel macrohomogeneity}
\protect\hypertarget{hill-mandel-macrohomogeneity}{}
\begin{itemize}
\tightlist
\item
  First we note that
\end{itemize}

\[\bar{\sigma}_{ij} \bar{\epsilon}_{ij} = \frac{1}{V} \int_V \sigma_{ij} \bar{\epsilon}_{ij} dV = \frac{1}{V} \int_V \bar{\sigma}_{ij} \epsilon_{ij} dV = \frac{1}{V} \int_V \bar{\sigma}_{ij} u_{i,j} dV\]

\begin{itemize}
\tightlist
\item
  Thus we can say that when
  \(\langle \sigma_{ij} \epsilon_{ij} \rangle = \bar{\sigma}_{ij} \bar{\epsilon}_{ij}\)
\end{itemize}

\[\langle \sigma_{ij} \epsilon_{ij} \rangle - \bar{\sigma}_{ij} \bar{\epsilon}_{ij} = \frac{1}{V} \int_V (\sigma_{ij}u_{i,j} - \bar{\sigma}_{ij} u_{i,j} - \sigma_{ij}\bar{\epsilon}_{ij}+\bar{\sigma}_{ij} \bar{\epsilon}_{ij})dV\]
\end{frame}

\begin{frame}{hill mandel macrohomogeneity}
\protect\hypertarget{hill-mandel-macrohomogeneity-1}{}
\begin{itemize}
\tightlist
\item
  After some algebra and applying the divergence theorem, we can write
  this as
\end{itemize}

\[\langle \sigma_{ij} \epsilon_{ij} \rangle - \bar{\sigma}_{ij} \bar{\epsilon}_{ij} = \frac{1}{V} \oint_V n_k(\sigma_{ik} - \bar{\sigma_{ik}})(u_{i} - x_j\bar{\epsilon}_{ij})dS\]

\begin{itemize}
\tightlist
\item
  The right-hand side can be made to vanish in various ways, but the
  most common are homogeneous traction, homogeneous displacement, and
  periodic boundary conditions
\end{itemize}
\end{frame}

\begin{frame}{finite elements}
\protect\hypertarget{finite-elements}{}
\begin{itemize}
\tightlist
\item
  There are a few things we need to do when using finite elements
\item
  First, we should ensure the mesh we use is periodic
\item
  Second, we should ensure that our boundary conditions satisfy
  Hill-Mandel and that our mesh is converged
\item
  Periodic boundary conditions converge more quickly than homogeneous
  stress or displacement
\item
  Third, we should repeat our periodic structure (2x2, 3x3) to check
  that the effective stiffness remains constant
\item
  We find homogenized properties by taking the volume-averaged stress
  and strain
\end{itemize}
\end{frame}

\end{document}
