% Options for packages loaded elsewhere
\PassOptionsToPackage{unicode}{hyperref}
\PassOptionsToPackage{hyphens}{url}
%
\documentclass[
  letterpaper,
  ignorenonframetext,
  aspectratio=43,
  handout,
  12pt]{beamer}
\usepackage{pgfpages}
\setbeamertemplate{caption}[numbered]
\setbeamertemplate{caption label separator}{: }
\setbeamercolor{caption name}{fg=normal text.fg}
\beamertemplatenavigationsymbolsempty
% Prevent slide breaks in the middle of a paragraph
\widowpenalties 1 10000
\raggedbottom
\setbeamertemplate{part page}{
  \centering
  \begin{beamercolorbox}[sep=16pt,center]{part title}
    \usebeamerfont{part title}\insertpart\par
  \end{beamercolorbox}
}
\setbeamertemplate{section page}{
  \centering
  \begin{beamercolorbox}[sep=12pt,center]{part title}
    \usebeamerfont{section title}\insertsection\par
  \end{beamercolorbox}
}
\setbeamertemplate{subsection page}{
  \centering
  \begin{beamercolorbox}[sep=8pt,center]{part title}
    \usebeamerfont{subsection title}\insertsubsection\par
  \end{beamercolorbox}
}
\AtBeginPart{
  \frame{\partpage}
}
\AtBeginSection{
  \ifbibliography
  \else
    \frame{\sectionpage}
  \fi
}
\AtBeginSubsection{
  \frame{\subsectionpage}
}
\usepackage{amsmath,amssymb}
\usepackage{lmodern}
\usepackage{ifxetex,ifluatex}
\ifnum 0\ifxetex 1\fi\ifluatex 1\fi=0 % if pdftex
  \usepackage[T1]{fontenc}
  \usepackage[utf8]{inputenc}
  \usepackage{textcomp} % provide euro and other symbols
\else % if luatex or xetex
  \usepackage{unicode-math}
  \defaultfontfeatures{Scale=MatchLowercase}
  \defaultfontfeatures[\rmfamily]{Ligatures=TeX,Scale=1}
\fi
\usetheme[]{metropolis}
% Use upquote if available, for straight quotes in verbatim environments
\IfFileExists{upquote.sty}{\usepackage{upquote}}{}
\IfFileExists{microtype.sty}{% use microtype if available
  \usepackage[]{microtype}
  \UseMicrotypeSet[protrusion]{basicmath} % disable protrusion for tt fonts
}{}
\makeatletter
\@ifundefined{KOMAClassName}{% if non-KOMA class
  \IfFileExists{parskip.sty}{%
    \usepackage{parskip}
  }{% else
    \setlength{\parindent}{0pt}
    \setlength{\parskip}{6pt plus 2pt minus 1pt}}
}{% if KOMA class
  \KOMAoptions{parskip=half}}
\makeatother
\usepackage{xcolor}
\IfFileExists{xurl.sty}{\usepackage{xurl}}{} % add URL line breaks if available
\IfFileExists{bookmark.sty}{\usepackage{bookmark}}{\usepackage{hyperref}}
\hypersetup{
  hidelinks,
  pdfcreator={LaTeX via pandoc}}
\urlstyle{same} % disable monospaced font for URLs
\newif\ifbibliography
% Make links footnotes instead of hotlinks:
\DeclareRobustCommand{\href}[2]{#2\footnote{\url{#1}}}
\setlength{\emergencystretch}{3em} % prevent overfull lines
\providecommand{\tightlist}{%
  \setlength{\itemsep}{0pt}\setlength{\parskip}{0pt}}
\setcounter{secnumdepth}{-\maxdimen} % remove section numbering
\usepackage{pgfpages}
\pgfpagesuselayout{2 on 1}
\providecommand{\tightlist}{%
\setlength{\itemsep}{0pt}\setlength{\parskip}{0pt}}
\makeatletter
\makeatother
\let\Oldincludegraphics\includegraphics
\renewcommand{\includegraphics}[2][]{\Oldincludegraphics[width=\textwidth,height=0.7\textheight,keepaspectratio]{#2}}
\ifluatex
  \usepackage{selnolig}  % disable illegal ligatures
\fi

\author{}
\date{}

\begin{document}

\begin{frame}
Lecture 9 - Variational Calculus

Dr.~Nicholas Smith

Wichita State University, Department of Aerospace Engineering

2 March, 2021
\end{frame}

\begin{frame}{schedule}
\protect\hypertarget{schedule}{}
\begin{itemize}
\tightlist
\item
  Mar 2 - Variational Calculus
\item
  Mar 4 - Boundary Conditions (HW3 Due)
\item
  Mar 9 - Project Descriptions
\item
  March 11 - SwiftComp
\end{itemize}
\end{frame}

\begin{frame}{outline}
\protect\hypertarget{outline}{}
\begin{itemize}
\tightlist
\item
  boundary conditions
\item
  multiple variables
\end{itemize}
\end{frame}

\begin{frame}{homework}
\protect\hypertarget{homework}{}
\begin{itemize}
\tightlist
\item
  My Python functions are not a substitute for understanding the math
\item
  You can program in any language, but it is also possible to do
  Mori-Tanaka in Excel
\item
  In my code I switched between tensor and matrix notation to avoid
  re-writing equations
\item
  Alternatively, we could re-write tensor equations entirely
\end{itemize}
\end{frame}

\begin{frame}{tensor equations}
\protect\hypertarget{tensor-equations}{}
\(a_{ijkl}^q = a_{ij}a_{kl}\)

\[\begin{gathered}
  a_4^l = -\frac{1}{35}(\delta_{ij} \delta_{kl} + \delta_{ik} \delta_{jl} + \delta_{il}\delta_{jk}) + \\
  \frac{1}{7} (a_{ij} \delta_{kl} + a_{ik}\delta_{jl} + a_{il}\delta_{jk} + a_{kl} \delta_{ij} + a_{jl}\delta_{ik} + a_{jk}\delta_{il})
\end{gathered}\]

\begin{itemize}
\tightlist
\item
  NOTE: Many of you copied my linear closure approximation, which used
  constants for 2D orientation
\item
  In 2D replace \(-\frac{1}{35}\) and \(\frac{1}{7}\) with
  \(-\frac{1}{24}\) and \(\frac{1}{6}\), respectively
\end{itemize}
\end{frame}

\hypertarget{boundary-conditions}{%
\section{boundary conditions}\label{boundary-conditions}}

\begin{frame}{boundaries}
\protect\hypertarget{boundaries}{}
\begin{itemize}
\tightlist
\item
  Not all problems of functionals have well-defined boundary conditions
\item
  The Euler-Lagrange equation will be the same
\item
  Consider the example
\end{itemize}

\[I[y] = \int_{x_0}^{x_1} [p(x)(\dot{y})^2 + q(x) y^2 + f(x)y]dx + h_1y^2(x_1) + h_0y^2(x_0)\]
\end{frame}

\begin{frame}{boundaries}
\protect\hypertarget{boundaries-1}{}
\begin{itemize}
\tightlist
\item
  For the functional to be stationary we have
\end{itemize}

\[\begin{gathered}
  I[y] = 2\int_{x_0}^{x_1} [-\dot{(p\dot{y})} + qy + f]\delta y dx + \\
  2p\dot{y} \delta y|_{x_0}^{x_1} + 2h_1y(x_1)\delta y(x_1) + 2h_0y(x_0)\delta y(x_0) = 0
\end{gathered}\]

\begin{itemize}
\tightlist
\item
  Satisfying the Euler-Lagrange equation will ensure the first line is
  equal to zero
\item
  The second line forms the natural boundary conditions
\end{itemize}

\[\begin{aligned}
  p(x_1) \dot{y}(x_1) + h_1 y(x_1) &= 0\\
  -p(x_0) \dot{y}(x_0) + h_0 y(x_0) &= 0
\end{aligned}\]
\end{frame}

\begin{frame}{natural and geometric boundaries}
\protect\hypertarget{natural-and-geometric-boundaries}{}
\begin{itemize}
\tightlist
\item
  In general, if a functional contains the derivative of an unknown
  function to the \emph{m}th order:
\item
  Boundary conditions expressed in terms of the unknown function to the
  (\emph{m}-1)th order are geometric boundary conditions
\item
  Boundary conditions expressed in terms of the unknown function higher
  than the (\emph{m} - 1)th order are natural boundary conditions
\item
  When there are geometric boundaries, the variation will be zero at the
  boundaries
\item
  Otherwise the coefficients must equal zero
\end{itemize}
\end{frame}

\begin{frame}{example}
\protect\hypertarget{example}{}
\begin{itemize}
\tightlist
\item
  Find the governing differential equation and boundary conditions for a
  bar of stiffness \emph{EA}, length \emph{L}
\item
  Subjected to a tensile load, \emph{p}(\emph{x})
\item
  There is a spring of stiffness \emph{k} attached to \emph{x}=\emph{L}
\item
  The bar is fixed at \emph{x}=0
\end{itemize}
\end{frame}

\begin{frame}{subsidiary conditions}
\protect\hypertarget{subsidiary-conditions}{}
\begin{itemize}
\tightlist
\item
  We have discussed problems with or without prescribed boundary
  conditions
\item
  We may also have additional constraints (also known as subsidiary
  conditions)
\item
  The can be formulated using the same method as the Lagrange Multiplier
\end{itemize}
\end{frame}

\begin{frame}{subsidiary conditions}
\protect\hypertarget{subsidiary-conditions-1}{}
\begin{itemize}
\tightlist
\item
  Consider a functional
\end{itemize}

\[I = \int_{x_0}^{x_1} F(y, \dot{y}, x) dx\]

\begin{itemize}
\tightlist
\item
  With boundary conditions, \(y(x_0)=y_0\) and \(y(x_1)=y_1\)
\item
  And the subsidiary condition
\end{itemize}

\[\int_{x_0}^{x_1} G(y, \dot{y}, x)dx = C\]
\end{frame}

\begin{frame}{subsidiary conditions}
\protect\hypertarget{subsidiary-conditions-2}{}
\begin{itemize}
\tightlist
\item
  The stationary conditions for this functional can be obtained using
\end{itemize}

\[\delta I^* = 0\]

\begin{itemize}
\tightlist
\item
  Where
\end{itemize}

\[I^* = \int_{x_0}^{x_1} F(y, \dot{y}, x) dx + \lambda\left(\int_{x_0}^{x_1} G(y, \dot{y}, x)dx - C\right)\]
\end{frame}

\begin{frame}{subsidiary conditions}
\protect\hypertarget{subsidiary-conditions-3}{}
\begin{itemize}
\tightlist
\item
  Carrying out the variation we find
\end{itemize}

\[\begin{gathered}
  \delta I^* = \int_{x_0}^{x_1} \left \{ \frac{\partial F}{\partial y} - \frac{d}{dx} \frac{\partial F}{\partial \dot{y}} + \lambda \left [ \frac{\partial G}{\partial y} - \frac{d}{dx} \frac{\partial G}{\partial \dot{y}} \right ] \right \} \delta y dx + \\
  \delta \lambda \left( \int_{x_0}^{x_1} G(y, \dot{y}, x)dx - C \right) = 0
\end{gathered}\]

\begin{itemize}
\tightlist
\item
  Which gives the Euler-Lagrange equation
\end{itemize}

\[\frac{\partial F}{\partial y} - \frac{d}{dx} \frac{\partial F}{\partial \dot{y}} + \lambda \left[ \frac{\partial G}{\partial y} - \frac{d}{dx} \frac{\partial G}{\partial \dot{y}} \right]\]
\end{frame}

\begin{frame}{subsidiary conditions}
\protect\hypertarget{subsidiary-conditions-4}{}
\begin{itemize}
\tightlist
\item
  If the subsidiary condition is given in terms of differential
  equations instead of an integral
\end{itemize}

\[G(x,y,\dot{y}) = 0\]

\begin{itemize}
\tightlist
\item
  Then we must write the functional as
\end{itemize}

\[J[y,\lambda] = \int_{x_0}^{x_1} F(y, \dot{y}, x) dx + \int_{x_0}^{x_1} \lambda G(y, \dot{y}, x)dx\]

\begin{itemize}
\tightlist
\item
  Since \(\lambda\) will be a function of \emph{x}
\end{itemize}
\end{frame}

\begin{frame}{subsidiary conditions}
\protect\hypertarget{subsidiary-conditions-5}{}
\begin{itemize}
\tightlist
\item
  The only difference in the Euler-Lagrange solution is that \(\lambda\)
  will be inside the derivative
\end{itemize}

\[\frac{\partial F}{\partial y} - \frac{d}{dx} \frac{\partial F}{\partial \dot{y}} + \lambda \frac{\partial G}{\partial y} - \frac{d}{dx} \left(\lambda \frac{\partial G}{\partial \dot{y}}\right)\]
\end{frame}

\begin{frame}{example}
\protect\hypertarget{example-1}{}
\begin{itemize}
\tightlist
\item
  A uniform power line with length \emph{C} and density \(\rho\) is
  hanging between two points, \((x_0,y_0)\) and \((x_1,y_2)\)
\item
  With gravity acting in the \emph{y} direction, find the shape of the
  power line in equilibrium
\end{itemize}
\end{frame}

\hypertarget{multiple-variables}{%
\section{multiple variables}\label{multiple-variables}}

\begin{frame}{higher derivatives}
\protect\hypertarget{higher-derivatives}{}
\begin{itemize}
\tightlist
\item
  While our development has only used one derivative of \emph{y}, it can
  easily be extended
\end{itemize}

\[I[y] = \int_{x_0}^x{1} F(x,y,\dot{y},\ddot{y},\dddot{y},...,y^{(n)}) dx\]

\begin{itemize}
\tightlist
\item
  The first variation is
\end{itemize}

\[\delta I[y] = \int_{x_0}^x{1} \left[ \frac{\partial F}{\partial y} \delta y + \frac{\partial F}{\partial \dot{y}}\delta \dot{y} + ... + \frac{\partial F}{\partial y^{(n)}} \delta y^{(n)}\right]dx\]

\begin{itemize}
\tightlist
\item
  Carrying out successive integration by parts we find
\end{itemize}

\[\delta I[y] = \int_{x_0}^x{1} \left[ \frac{\partial F}{\partial y} - \frac{d}{dx}\left( \frac{\partial F}{\partial \dot{y}}\right) + ... + (-1)^n \frac{d^{n}}{dx^{n}}\left(\frac{\partial F}{\partial y^{(n)}}\right)\right] \delta ydx\]
\end{frame}

\begin{frame}{higher derivatives}
\protect\hypertarget{higher-derivatives-1}{}
\begin{itemize}
\tightlist
\item
  The Euler-Lagrange equation is merely in the terms inside the integral
\item
  Boundary terms from integration vanish when
  \(y, \dot{y}, ... , y^{(n)}\) are prescribed at the boundaries
\end{itemize}
\end{frame}

\begin{frame}{multiple functions}
\protect\hypertarget{multiple-functions}{}
\begin{itemize}
\tightlist
\item
  A functional could also consist of several functions, for example
\end{itemize}

\[I[y,z] = \int_{x_0}^x{1} F(x,y,z,\dot{y},\dot{z})dx\]

\begin{itemize}
\tightlist
\item
  Where both \emph{y} and \emph{z} are functions of \emph{x}
\item
  In this case the Euler-Lagrange equation is two equations
\end{itemize}

\[\frac{\partial F}{\partial y} - \frac{d}{dx}\left(\frac{\partial F}{\partial \dot{y}}\right) = 0 \qquad \frac{\partial F}{\partial z} - \frac{d}{dx}\left(\frac{\partial F}{\partial \dot{z}}\right) = 0\]
\end{frame}

\begin{frame}{multiple variables}
\protect\hypertarget{multiple-variables-1}{}
\begin{itemize}
\tightlist
\item
  We could also have multiple fundamental variables in the functional,
  for example
\end{itemize}

\[I [u] = \int \int_G F(x, y, u, u_{,x}, u_{,y}) dx dy\]

\begin{itemize}
\tightlist
\item
  The Euler-Lagrange equation is
\end{itemize}

\[\frac{\partial F}{\partial u} - \frac{\partial}{\partial x} \left( \frac{\partial F}{\partial u_{,x}} \right) - \frac{\partial}{\partial y} \left( \frac{\partial F}{\partial u_{,y}}\right) = 0\]

\begin{itemize}
\tightlist
\item
  If \emph{u} is prescribed along the boundary, then \(\delta u = 0\)
  along the boundary, otherwise
\end{itemize}

\[\frac{\partial F}{\partial u_{,x}} n_x + \frac{\partial F}{\partial u_{,y}} n_y = 0\]

along the boundary
\end{frame}

\begin{frame}{example}
\protect\hypertarget{example-2}{}
\begin{itemize}
\tightlist
\item
  Minimize the mechanical potential energy of a beam with deflection
  \emph{y} under applied force, \emph{f}(\emph{x})
\end{itemize}

\[I[y] = \int_0^L \left[ \frac{1}{2} EI (\ddot{y})^2 - fy\right] dx\]
\end{frame}

\begin{frame}{example}
\protect\hypertarget{example-3}{}
\begin{itemize}
\tightlist
\item
  Minimize the functional
\end{itemize}

\[I[y,z] = \int_{x_0}^{x_1} (y^2 - z^2) dx \]

\begin{itemize}
\tightlist
\item
  Under the constraint
\end{itemize}

\[\dot{y} - y + z = 0\]
\end{frame}

\begin{frame}{next class}
\protect\hypertarget{next-class}{}
\begin{itemize}
\tightlist
\item
  Converting between differential and variational statements
\item
  Approximate solutions
\item
  Variational asymptotic method
\end{itemize}
\end{frame}

\end{document}
